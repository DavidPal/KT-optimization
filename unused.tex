\section{Unused}
We consider algorithms of certain form, which we call betting algorithms.  A
betting algorithm starts with $\Wealth_0 = 1$ dollars.
In every round, it chooses a ``fraction'' $\beta_t \in \H$, $\|\beta\| \le 1$
of its current wealth to bet. The algorithm predicts
$$
w_t = \beta_t \cdot \Wealth_{t-1}
$$
where $\Wealth_t = 1 + \sum_{i=1}^t \langle g_t, w_t \rangle$. After making its prediction,
the algorithm's wealth is
$$
\Wealth_t = \langle g_t, w_t \rangle + \Wealth_{t-1} = \langle g_t, \beta_t \rangle \Wealth_{t-1} + \Wealth_t = (1 + \langle g_t, \beta_t \rangle) \Wealth_{t-1} \; .
$$
Hence
$$
\Wealth_T = \prod_{t=1}^T (1 + \langle g_t, \beta_t \rangle) \; .
$$


\begin{theorem}
\label{theo:wealth}
For any sequence of $g_1, g_2, \dots, g_T \in \H$ such that $\|g_1\|, \|g_2\|,
\dots, \|g_T\| \le 1$, the algorithm predicting ``fraction'' $\beta_t =
\frac{1}{t} \sum_{i=1}^{t-1} g_i$ satisfies
\begin{align*}
\Wealth_T
&\geq \frac{2^T \cdot \Gamma \left(\frac{T+1}{2} + \frac{1}{2}\left\|\sum_{t=1}^T g_t \right\| \right) \cdot \Gamma \left(\frac{T+1}{2} - \frac{1}{2} \left\|\sum_{t=1}^T g_t \right\| \right)}{\pi \cdot T!} \\
&\geq \exp\left(T \, \DB\left(\frac{1}{2} +\frac{\left\|\sum_{t=1}^T g_t \right\|}{2T}\middle\| \frac{1}{2}\right)\right)
\; .
\end{align*}
\end{theorem}
%
\begin{proof}
We prove the inequality by induction on $T$. In the base case $T=0$, the
left-hand side is $\Wealth_0 = 1$, and the right-hand side equals $1$ as well
since $\sum_{t=1}^T g_t = 0$ and $\Gamma(\frac{1}{2}) = \sqrt{\pi}$. For $T \ge
1$, we will show that the difference of left and right side of the inequality is
non-negative. We have
\begin{align*}
& \Wealth_T \ - \ \frac{2^T \cdot \Gamma \left(\frac{T+1}{2} + \frac{1}{2}\left\|\sum_{t=1}^T g_t \right\| \right) \cdot \Gamma \left(\frac{T+1}{2} - \frac{1}{2} \left\|\sum_{t=1}^T g_t \right\| \right)}{\pi \cdot T!} \\
& = (1 + \langle g_T, \beta_T \rangle) \Wealth_{T-1} \ - \ \frac{2^T \cdot \Gamma \left(\frac{T+1}{2} + \frac{1}{2}\left\|\sum_{t=1}^T g_t \right\| \right) \cdot \Gamma \left(\frac{T+1}{2} - \frac{1}{2} \left\|\sum_{t=1}^T g_t \right\| \right)}{\pi \cdot T!} \\
& = \Wealth_{T-1} \ + \ \frac{1}{T} \cdot \Wealth_{T-1} \cdot \left(\langle g_T, T \beta_T \rangle - \frac{2^T T \cdot \Gamma \left(\frac{T+1}{2} + \frac{1}{2}\left\|\sum_{t=1}^T g_t \right\| \right) \cdot \Gamma \left(\frac{T+1}{2} - \frac{1}{2} \left\|\sum_{t=1}^T g_t \right\| \right)}{\Wealth_{T-1} \cdot \pi \cdot T!} \right) \\
& = \Wealth_{T-1} \ + \ \frac{1}{T} \cdot \Wealth_{T-1} \cdot \left( \left\langle g_T, \sum_{t=1}^{T-1} g_t \right\rangle + h\left( \left\|g_T + \sum_{t=1}^{T-1} g_t \right\| \right) \right) \; .
\end{align*}
where we define
\begin{align*}
h(x) & = - C \cdot \Gamma \left(\frac{T+1}{2} + \frac{x}{2} \right) \cdot \Gamma \left(\frac{T+1}{2} - \frac{x}{2} \right) \; ,  \\
C & = \frac{2^T \cdot T}{\Wealth_{T-1} \cdot \pi \cdot T!} \; .
\end{align*}
We apply Lemma~\ref{lemma:extremes} to $h(x)$. First note that $C > 0$. Second,
$\Gamma(x)$ is logarithmically convex on $[0,\infty)$, and hence $h(x)$ is
logarithmically concave and hence concave. Third, $h(x)$ is clearly even and its
defined on an interval $(-\frac{T+1}{2}, \frac{T+1}{2})$. To see that $h(x)$
satisfies the condition $x \cdot h''(x) \le h'(x)$, we apply
Lemma~\ref{lemma:gamma-function}. Let $f(x) = \Gamma(\frac{T+1}{2}+x)
\Gamma(\frac{T+1}{2}-x)$. By Lemma~\ref{lemma:gamma-function}, we have $x \cdot
f''(x) \ge f(x)$ for all $x \in [0,a)$. Since $h(x) = - C f(x/2)$, we have $x
\cdot h''(x) \le h'(x)$ for $x \in [0,T+1)$.

Therefore, since $\|g_t\| \le 1$ and $\|\sum_{t=1}^{T-1} g_t\| < T$,
\begin{align*}
& \Wealth_{T-1} \ + \ \frac{1}{T} \cdot \Wealth_{T-1} \cdot \left( \left\langle g_T, \sum_{t=1}^{T-1} g_t \right\rangle + h\left( \left\|g_T + \sum_{t=1}^{T-1} g_t \right\| \right) \right) \\
& \ge \Wealth_{T-1} \ + \ \frac{1}{T} \cdot \Wealth_{T-1} \cdot \min \left\{ \left\| \sum_{t=1}^{T-1} g_t \right\| + h\left( \left\|\sum_{t=1}^{T-1} g_t \right\| + 1 \right), - \left\| \sum_{t=1}^{T-1} g_t \right\| + h\left(\left\|\sum_{t=1}^{T-1} g_t \right\| - 1 \right) \right\} \\
& = \Wealth_{T-1} \ + \ \Wealth_{T-1} \cdot \min \left\{ \| \beta_T \| + \frac{1}{T} \cdot h\left( \left\|\sum_{t=1}^{T-1} g_t \right\| + 1 \right), - \|\beta_T\| + \frac{1}{T} \cdot h\left(\left\|\sum_{t=1}^{T-1} g_t \right\| - 1 \right) \right\} \\
& = \min \left\{ \Wealth_{T-1} (1 + \| \beta_T \|) + \frac{\Wealth_{T-1}}{T} h\left( \left\|\sum_{t=1}^{T-1} g_t \right\| + 1 \right), \Wealth_{T-1} (1 - \| \beta_T \|) + \frac{\Wealth_{T-1}}{T} h\left(\left\|\sum_{t=1}^{T-1} g_t \right\| - 1 \right) \right\}
\end{align*}
It remains to prove that the last expression is non-negative, which we do by
proving that each of the two sub-expressions of the minima are non-negative. The
first sub-expression is
\begin{align*}
& \Wealth_{T-1} (1 + \| \beta_T \|) + \frac{\Wealth_{T-1}}{T} h\left( \left\|\sum_{t=1}^{T-1} g_t \right\| + 1 \right) \\
& =  \Wealth_{T-1} (1 + \|\beta_T\|) - \frac{2^T \cdot \Gamma \left(\frac{T+1}{2} + \frac{1}{2}\left\|\sum_{t=1}^{T-1} g_t \right\| + \frac{1}{2} \right) \cdot \Gamma \left(\frac{T+1}{2} - \frac{1}{2} \left\|\sum_{t=1}^{T-1} g_t \right\| - \frac{1}{2} \right)}{\pi \cdot T!} \\
& =  \Wealth_{T-1} (1 + \|\beta_T\|) - \frac{2^T \left( \frac{T}{2} + \frac{1}{2} \left\|\sum_{t=1}^{T-1} g_t \right\| \right) \Gamma \left(\frac{T}{2} + \frac{1}{2}\left\|\sum_{t=1}^{T-1} g_t \right\| \right) \cdot \Gamma \left(\frac{T}{2} - \frac{1}{2} \left\|\sum_{t=1}^{T-1} g_t \right\| \right)}{\pi \cdot T!} \\
& =  \Wealth_{T-1} (1 + \|\beta_T\|) - \frac{2^{T-1} \left( 1 + \|\beta_T\| \right) \Gamma \left(\frac{T}{2} + \frac{1}{2}\left\|\sum_{t=1}^{T-1} g_t \right\| \right) \cdot \Gamma \left(\frac{T}{2} - \frac{1}{2} \left\|\sum_{t=1}^{T-1} g_t \right\| \right)}{\pi \cdot (T-1)!} \\
& =  (1 + \|\beta_T\|) \left( \Wealth_{T-1} - \frac{2^{T-1} \Gamma \left(\frac{T}{2} + \frac{1}{2}\left\|\sum_{t=1}^{T-1} g_t \right\| \right) \cdot \Gamma \left(\frac{T}{2} - \frac{1}{2} \left\|\sum_{t=1}^{T-1} g_t \right\| \right)}{\pi \cdot (T-1)!} \right) \; .
\end{align*}
The second sub-expression is
\begin{align*}
& \Wealth_{T-1} (1 - \| \beta_T \|) + \frac{\Wealth_{T-1}}{T} h\left(\left\|\sum_{t=1}^{T-1} g_t \right\| - 1 \right) \\
& = \Wealth_{T-1} (1 - \|\beta_T\|) - \frac{2^T \cdot \Gamma \left(\frac{T+1}{2} + \frac{1}{2}\left\|\sum_{t=1}^{T-1} g_t \right\| - \frac{1}{2} \right) \cdot \Gamma \left(\frac{T+1}{2} - \frac{1}{2} \left\|\sum_{t=1}^{T-1} g_t \right\| + \frac{1}{2} \right)}{\pi \cdot T!} \\
& = \Wealth_{T-1} (1 - \|\beta_T\|) - \frac{2^T \left( \frac{T}{2} - \frac{1}{2}\|\sum_{t=1}^{T-1} g_t\| \right) \cdot \Gamma \left(\frac{T}{2} + \frac{1}{2}\left\|\sum_{t=1}^{T-1} g_t \right\| \right) \cdot \Gamma \left(\frac{T}{2} - \frac{1}{2} \left\|\sum_{t=1}^{T-1} g_t \right\| \right)}{\pi \cdot T!} \\
& = \Wealth_{T-1} (1 - \|\beta_T\|) - \frac{2^{T-1} \left( 1 - \|\beta_T\| \right) \cdot \Gamma \left(\frac{T}{2} + \frac{1}{2}\left\|\sum_{t=1}^{T-1} g_t \right\| \right) \cdot \Gamma \left(\frac{T}{2} - \frac{1}{2} \left\|\sum_{t=1}^{T-1} g_t \right\| \right)}{\pi \cdot (T-1)!} \\
& = (1 - \|\beta_T\|) \left( \Wealth_{T-1}  - \frac{2^{T-1} \Gamma \left(\frac{T}{2} + \frac{1}{2}\left\|\sum_{t=1}^{T-1} g_t \right\| \right) \cdot \Gamma \left(\frac{T}{2} - \frac{1}{2} \left\|\sum_{t=1}^{T-1} g_t \right\| \right)}{\pi \cdot (T-1)!} \right) \; .
\end{align*}
In both cases we have used $\Gamma(x+1) = x \cdot \Gamma(x)$ which holds for any
real $x > 0$. Induction hypothesis and $\|\beta_T\| \in (-1,1)$ imply
that both sub-expressions are non-negative.
\end{proof}


\section{Proof of of Lemma~\ref{lemma:kt} and Lemma Theorem~\ref{theo:logloss}}

\begin{proof}[Proof of Lemma~\ref{lemma:kt}]
We prove the equality by induction $T$. For $T=0$ the equality is
$$
0 = - \ln \left( \frac{\Gamma(1/2) \cdot \Gamma(1/2)}{\pi \cdot 0!} \right)
$$
which holds true since $\Gamma(1/2) = \sqrt{\pi}$.
For $T \ge 1$, we use the induction hypothesis for $T-1$ and some algebraic manipulation
\begin{align*}
\sum_{t=1}^T \ell(p_t, q_t)
& = \ell(p_T, q_T) + \sum_{t=1}^{T-1} \ell(p_t, q_t) \\
& = \ell(p_T, q_T) - \ln \left( \frac{\Gamma(a_{T-1} + 1/2) \cdot \Gamma(b_{T-1} + 1/2)}{\pi \cdot (T-1)!} \right) \\
& = q_T \ln\left( \frac{1}{p_T}\right) + (1-q_T) \ln  \left( \frac{1}{1 - p_T} \right) - \ln \left( \frac{\Gamma(a_{T-1} + 1/2) \cdot \Gamma(b_{T-1} + 1/2)}{\pi \cdot (T-1)!} \right) \\
& = - q_T \ln\left( \frac{\frac{1}{2} + a_{T-1}}{T} \right) - (1-q_T) \ln\left( \frac{\frac{1}{2} + b_{T-1}}{T} \right) - \ln \left( \frac{\Gamma(a_{T-1} + 1/2) \cdot \Gamma(b_{T-1} + 1/2)}{\pi \cdot (T-1)!} \right) \\
& = - \ln\left( \left( \frac{\frac{1}{2} + a_{T-1}}{T} \right)^{q_T} \left( \frac{\frac{1}{2} + b_{T-1}}{T} \right)^{1-q_T} \frac{\Gamma(a_{T-1} + 1/2) \cdot \Gamma(b_{T-1} + 1/2)}{\pi \cdot (T-1)!} \right) \\
& = - \ln\left( \left( \frac{1}{2} + a_{T-1} \right)^{q_T} \left( \frac{1}{2} + b_{T-1} \right)^{1-q_T} \frac{\Gamma(a_{T-1} + 1/2) \cdot \Gamma(b_{T-1} + 1/2)}{\pi \cdot T!}  \right) \; .
\end{align*}
It remains to prove that expression inside logarithm equals
$$
\frac{\Gamma(a_{T} + 1/2) \cdot \Gamma(b_T + 1/2)}{\pi \cdot T!}
$$
We consider two cases. If $q_T = 1$ then
\begin{align*}
& \left( \frac{1}{2} + a_{T-1} \right)^{q_T} \left( \frac{1}{2} + b_{T-1} \right)^{1-q_T} \frac{\Gamma(a_{T-1} + 1/2) \cdot \Gamma(b_{T-1} + 1/2)}{\pi \cdot T!} \\
& = \left( \frac{1}{2} + a_{T-1} \right) \frac{\Gamma(a_{T-1} + 1/2) \cdot \Gamma(b_{T-1} + 1/2)}{\pi \cdot T!} \\
& = \frac{\Gamma(a_{T-1} + 3/2) \cdot \Gamma(b_{T-1} + 1/2)}{\pi \cdot T!} \\
& = \frac{\Gamma(a_{T} + 1/2) \cdot \Gamma(b_T + 1/2)}{\pi \cdot T!} \; ,
\end{align*}
where we have used that $\Gamma(x+1) = x \Gamma(x)$ for any real $x > 0$ and that $a_T = a_{T-1} + q_T = a_{T-1} + 1$ and $b_T = b_{T-1} + (1-q_T) = b_{T-1}$.
Similarly, if $q_T = 0$ then
\begin{align*}
& \left( \frac{1}{2} + a_{T-1} \right)^{q_T} \left( \frac{1}{2} + b_{T-1} \right)^{1-q_T} \frac{\Gamma(a_{T-1} + 1/2) \cdot \Gamma(b_{T-1} + 1/2)}{\pi \cdot T!} \\
& = \left( \frac{1}{2} + b_{T-1} \right) \frac{\Gamma(a_{T-1} + 1/2) \cdot \Gamma(b_{T-1} + 1/2)}{\pi \cdot T!} \\
& = \frac{\Gamma(a_{T-1} + 1/2) \cdot \Gamma(b_{T-1} + 3/2)}{\pi \cdot T!} \\
& = \frac{\Gamma(a_T + 1/2) \cdot \Gamma(b_T + 3/2)}{\pi \cdot T!} \; ,
\end{align*}
where we have used that $\Gamma(x+1) = x \Gamma(x)$ for any real $x > 0$ and that $a_T = a_{T-1} + q_T = a_{T-1}$ and $b_T = b_{T-1} + (1-q_T) = b_{T-1} + 1$.
\end{proof}


\subsection{KT estimator for $\beta_t$}

We use KT estimator to define fractions $\beta_t$. Let
\begin{align*}
q_t & = \frac{1 + g_t}{2} \\
a_T & = \sum_{t=1}^T q_t \; , \\
b_T & = \sum_{t=1}^T (1 - q_t) = T - a_T \; , \\
p^* & =  \argmin_{p \in [0,1]} \sum_{t=1}^T \ell(p, q_t)
      = \frac{a_T}{a_T + b_T}
      = \frac{a_T}{T} = \frac{\sum_{t=1}^T q_t}{T} \; .
\end{align*}
We use the corresponding KT estimator
$$
p_t = \frac{\frac{1}{2} + \sum_{t=1}^{t-1} q_i}{t} = \frac{\frac{1}{2} + a_{t-1}}{t} = \frac{1}{2} + \frac{1}{2t} \sum_{i=1}^{t-1} g_i
$$
to define $\beta_t$ as
$$
\beta_t = 2p_t - 1 = \frac{1}{t} \sum_{i=1}^{t-1} g_i \; .
$$

We lower bound logarithm of the wealth as follows
\allowdisplaybreaks
\begin{align*}
\ln \Wealth_T
& = \ln \prod_{t=1}^T (1 + g_t\beta_t) \\
& =  \sum_{t=1}^T \ln (1 + g_t\beta_t) \\
& \ge  \sum_{t=1}^T \left( \frac{1+g_t}{2} \right) \ln \left(1 + \beta_t\right) + \left( \frac{1-g_t}{2} \right) \ln \left(1 - \beta_t \right) \\
& =  \sum_{t=1}^T \left( \frac{1+g_t}{2} \right) \ln \left(2p_t \right) + \left( \frac{1-g_t}{2} \right) \ln \left(2 (1 - p_t) \right) \\
& =  T \ln(2) + \sum_{t=1}^T \left( \frac{1+g_t}{2} \right) \ln (p_t) + \left( \frac{1-g_t}{2} \right) \ln (1 - p_t) \\
& =  T \ln(2) - \sum_{t=1}^T \ell(p_t, q_t) \\
& \ge  T \ln(2) - \ln(2) - \frac{1}{2} \ln(T) - T \cdot H(p^*) \\
& =  - \ln(2) - \frac{1}{2} \ln(T) - T \cdot (H(p^*) + \ln(1/2)) \\
& =  - \ln(2) - \frac{1}{2} \ln(T) + T \cdot D\left(p^*, 1/2 \right)
\end{align*}
where $D(u,v) = u \ln(u/v) + (1-v) \ln((1-u)/(1-v))$ is the Kullback-Leibler divergence
between Bernoulli random variables with parameters $u,v$.




\begin{proof}[Proof of Theorem~\ref{theo:logloss}]
The formulas for $a_T, b_T$ and $p^*$ remain exactly the same as in Section~\ref{section:online-data-compression}:
\begin{align*}
a_T & = \sum_{t=1}^T q_t \; , \\
b_T & = \sum_{t=1}^T (1 - q_t) = T - a_T \; , \\
p^* & =  \argmin_{p \in [0,1]} \sum_{t=1}^T \ell(p, q_t) = \frac{a_T}{a_T + b_T} = \frac{a_T}{T} = \frac{\sum_{t=1}^T q_t}{T} \; .
\end{align*}
Hence, we have $p_t = \frac{\frac{1}{2} + \sum_{t=1}^{t-1} q_i}{t} = \frac{\frac{1}{2} + a_{t-1}}{t}$.

Let $\widetilde q_1, \widetilde q_2, \dots, \widetilde q_T$ independent
Bernoulli variables with parameters $q_1, q_2, \dots, q_T$ respectively.
Let
\begin{align*}
\widetilde a_t & = \sum_{i=1}^t \widetilde q_i  \; , \\
\widetilde b_t & = t - \widetilde a_t \; , \\
\widetilde p_t & = \frac{\frac{1}{2} + \sum_{i=1}^{t-1} \widetilde q_i}{t} = \frac{\frac{1}{2} + \widetilde a_{t-1}}{t} \; , \\
\widehat p & = \frac{\widetilde a_T}{T} = \frac{\widetilde a_T}{\widetilde a_T + \widetilde b_T} \; . \\
\end{align*}
Clearly,
\begin{align*}
q_i & = \Exp[\widetilde q_i] \; , \\
a_t & = \Exp[\widetilde a_t] \; , \\
b_t & = \Exp[\widetilde b_t] \; , \\
p_t & = \Exp[\widetilde p_t] \; , \\
p^* & = \Exp[\widehat p] \; .
\end{align*}

Since $\ell(p,q)$ is linear in $q$ and convex in $p$, we have
\begin{align*}
\sum_{t=1}^T \ell(p_t, q_t)
& = \sum_{t=1}^T \ell( \Exp[ \widetilde p_t], \Exp[\widetilde q_t]) \\
& = \Exp\left[ \sum_{t=1}^T \ell( \Exp[ \widetilde p_t], \widetilde q_t) \right] \\
& \le \Exp\left[ \sum_{t=1}^T \ell( \widetilde p_t, \widetilde q_t) \right] \\
& \le - \Exp\left[ \ln \left\{ \frac{1}{2\sqrt{T}} \left( \frac{\widetilde a_T}{T} \right)^{\widetilde a_T} \left( \frac{\widetilde b_T}{T} \right)^{\widetilde b_T} \right\} \right] \\
& = \ln(2) + \frac{1}{2} \ln(T) + \Exp \left[ - \widetilde a_T \ln \left( \frac{\widetilde a_T}{T} \right) - \widetilde b_T \ln \left( \frac{\widetilde b_T}{T} \right) \right] \\
& = \ln(2) + \frac{1}{2} \ln(T) + T \cdot \Exp \left[ H(\widehat p) \right] \\
& \le \ln(2) + \frac{1}{2} \ln(T) + T \cdot H(\Exp[\widehat p]) \\
& = \ln(2) + \frac{1}{2} \ln(T) + T \cdot H(p^*)~.
\end{align*}
\end{proof}