\section{A Betting and Expert Algorithm for Known $T$}

In this section we show how it is straightforward to obtain new parameter-free expert algorithms with quantile guarantees.
As seen in the previous section, we need a 1-dimensional algorithm that has an exponential gain, up to negative logarithm terms. In alternative, given the duality between regret and wealth, we can look for 1-dimensional \ac{OLO} algorithms that have a $\scO(|u|\sqrt{T \log(\sqrt{T} |u|)})$ regret.

The \ac{KT} bettor suffers from the problem of having a negative logarithmic term in the exponent of the wealth function, that becomes a $log(T)$ factor in the expert reduction.
To fix it, it would be enough to add such a term in the potential function $f$. When the number of rounds is known beforehand, this can be easily done.

Hence, we present a betting algorithm for the case that the number of rounds $T$ is known. We show that in this case it is possible to have a quantile bound without spurios $\ln T$ or $\ln \ln T$.
We will use the following potential function
\begin{equation}
\label{eq:known_t}
f_t(x)=\epsilon_0 \exp\left(\frac{x^2}{T+t}-\ln\left(1+\frac{t}{T}\right)\right),
\end{equation}
with associated betting function
\[
b_t=2 S\left(\frac{4 \sum_{i=1}^{t-1} g_i}{T+t}\right)-1,
\]
where $S(x)=\frac{1}{1+\exp(-x)}$ is the classic sigmoidal function. Note that the sigmoidal function is just squashing the extreme value of
$\frac{\sum_{i=1}^{t-1} g_i}{T+t}$ to avoid to bet a large amount of money.

This function satisfies the first two conditions of Assumption~1.
We have to prove the third one.

\begin{theorem}
\label{theo:known_t}
The function in \eqref{eq:known_t} satisfies the third condition in Assumption~1.
\end{theorem}


Hence, using the above betting strategy, we have the following guarantee
\[
\Wealth_T = \epsilon_0 + \sum_{t=1}^T w_t g_t \geq \epsilon_0 \exp\left( \frac{(\sum_{t=1}^T g_t)^2}{2 T} - \ln 2\right),
\]
where $T$ is parameter of the algorithm.

Using this algorithm in Theorem~\ref{theo:expert_reduction}, we obtain a very clean quantile guarantee.
\begin{cor}
\label{cor:kt_expert}
Let $l_t$ an arbitrary sequence of loss vectors in $[0,1]^d$. Using the notation the Normal Potential as a base learner, where $\epsilon_0=1$, the following holds
\[
\sum_{t=1}^T \langle p_t, l_t\rangle -\sum_{t=1}^T \langle u , l_t\rangle 
\leq \sqrt{2 T \left(\DKL(u||q) +\ln 2 \right)} \; .
\]
\end{cor}
