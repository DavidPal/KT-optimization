\section{KT bettor}

\ac{KT} estimator is a classical estimator of bias of coin.
Given $T$ outcomes of a coin flip $q_1, q_2, \dots, q_T \in \{0,1\}$ where $1$
denotes heads and $0$ tails, KT estimator of coin's probability of heads is
$$
\frac{\frac{1}{2} + \sum_{t=1}^T q_t}{T + 1} \; .
$$
This estimator has many interesting properties. For us, the most crucial one is
that it can be used for online maximization of the wealth.

%First, in section~\ref{section:online-data-compression}, we recap the online
%data compression problem and show a bound on number of bits used by the KT
%estimator.  Second, in section~\ref{section:log-loss}, we generalize the online
%data compression problem to an online prediction with log loss and we show that
%a obvious generalization of the KT estimator yields and algorithm with the same
%bound. In section~\ref{section:online-linear-optimization-1D}, we explain the
%connection with one-dimensional online linear optimization problem.
%Finally, in section~\ref{section:online-linear-optimization-hilbert-space},
%we generalize it to online linear optimization over a Hilbert space.

\subsection{Online Log Wealth Maximization}
\label{section:online-data-compression}

Consider an online data compression problem where we predict a stream of bits;
in each round we predict the next bit based on the previous bits.  More
formally, in each round $t=1,2,\dots$, we predict probability $p_t \in [0,1]$
that the next bit equals one, and then the next bit $q_t \in \{0, 1\}$ is
observed.  Our prediction $p_t$ is scored according to log-loss:
$$
\ell(p_t, q_t) = q_t \ln\left( \frac{1}{p_t}\right) + (1-q_t) \ln  \left( \frac{1}{1 - p_t} \right) \; .
$$
In information theoretic terms, $\ell(p_t, q_t)$ measures the number of bits
needed to encode $q_t$ if we use encoding that uses $\ln(\frac{1}{p_t})$ bits
to encode $+1$ and $\ln(\frac{1}{1 - p_t})$ bits to encode $0$.

After $T$ rounds, the algorithm incurs the loss $\sum_{t=1}^T \ell(p_t, q_t)$. We
compare this loss with an offline baseline that in round $1,2,\dots,T$ predicts
the same prediction
$$
p^* = \argmin_{p \in [0,1]} \sum_{t=1}^T \ell(p, q_t) \; .
$$
Note that $p^*$ can be calculated only with the knowledge of all the outcomes
$q_1, q_2, \dots, q_T$; hence the name \emph{offline}.

Let
$$
a_T = \sum_{t=1}^T q_t \qquad \text{and} \qquad b_T = \sum_{t=1}^T (1 - q_t) = T - a_T \; .
$$
We have
\begin{align*}
p^*
= \argmin_{p \in [0,1]} \ a_T \cdot  \ln \left( \frac{1}{p} \right) + b_T \cdot \ln \left( \frac{1}{1-p} \right)
= \argmin_{p \in [0,1]} \ - a_T \cdot  \ln p \ - \  b_T \cdot \ln (1-p)
\end{align*}
Defining $f(p) = - a_T \cdot  \ln p \ - \  b_T \cdot \ln(1-p)$, we can find maximum of
finding point the derivative of $f(p)$ is zero. The derivative of $f(p)$ is
$$
f'(p) = - \frac{a_T}{p} + \frac{b_T}{1-p} \; .
$$
The equation $f'(p) = 0$ has only one solution
$$
p^* = \frac{a_T}{a_T + b_T} = \frac{a_T}{T} = \frac{\sum_{t=1}^T q_t}{T} \; .
$$
Note that $p^*$ differs from the KT estimate $p_T$.

The type of result that we are interested is an upper bound on difference
between the loss of the algorithm and the loss of the baseline
$$
\sum_{t=1}^T \ell(p_t, q_t) - \sum_{t=1}^T \ell(p^*, q_t) \; ,
$$
which can be viewed as the cost for not knowing the fraction of bits set to $1$
ahead of time.

%\subsection{KT estimator}

We consider an algorithm that in round $t$ chooses its prediction $p_t$
according to KT estimator
$$
p_t = \frac{\frac{1}{2} + \sum_{t=1}^{t-1} q_i}{t} = \frac{\frac{1}{2} + a_{t-1}}{t} \; .
$$

A little known result by \citep{KrichevskyT81} (see also ~\cite{Cesa-BianchiL06}), proves the following surprising \emph{equality}.
For completeness, its proofs and all the other proofs of this paper, are in the Appendix.
\begin{lemma}[Log loss of the KT Estimator]
\label{lemma:kt}
For any sequence $q_1, q_2, \dots, q_T \in \{0,1\}$, the log-loss of KT
predictions is
$$
\sum_{t=1}^T \ell(p_t, q_t) =  - \ln \left( \frac{\Gamma(a_T + 1/2) \cdot \Gamma(b_T + 1/2)}{\pi \cdot T!} \right)
$$
where $a_T = \sum_{t=1}^T q_t$ and $b_T = T - a_T$ and $\Gamma(x) =
\int_0^\infty x^{u-1} e^{-x} du$ is the Euler's Gamma function.
\end{lemma}

% We also consider a generalization of the previous game where $q_t$ are not
% restricted the set $\{0,1\}$ but they are allowed to be any real number in the
% interval $[0,1]$.
% 
% The KT estimator is generalized in the obvious way
% $$
% p_t = \frac{\frac{1}{2} + \sum_{t=1}^{t-1} q_i}{t}\; .
% $$
% 
% In this setting we can prove the following Theorem.
% \begin{theorem}
% \label{theo:logloss}
% For any sequence $q_1, q_2, \dots, q_T \in [0,1]$, the log-loss of KT
% predictions is
% $$
% %\sum_{t=1}^T \ell(p_t, q_t) \leq \ln(2) + \frac{1}{2} \ln(T) + T \cdot H(p^*)
% \sum_{t=1}^T \ell(p_t, q_t) \leq - \ln \left( \frac{\Gamma(a_T + 1/2) \cdot \Gamma(b_T + 1/2)}{\pi \cdot T!} \right) \; .
% $$
% \end{theorem}
% Notice that this theorem is exactly the same as before, the only difference ???


Also, it is easy to show the following inequality.
\begin{lemma}
\label{lemma:approx_gamma}
If $T$ is a non-negative integer and $a,b$ are non-negative reals such that $a +
b = T$ then
$$
\ln \frac{\Gamma(a + 1/2) \cdot \Gamma(b + 1/2)}{\pi \cdot T!} 
\geq -\ln(2) -\frac{1}{2} \ln(T) +\ln \left(\left( \frac{a}{T} \right)^a \left( \frac{b}{T} \right)^b\right) 
= -\ln(2) -\frac{1}{2} \ln(T) -T \ln (2) + T \, \DKL\left(\frac{b}{T}\middle\|\frac{1}{2}\right)\; .
$$
\end{lemma}

Combining these two lemmas we get
\begin{align*}
\sum_{t=1}^T \ell(p_t, q_t)
& \le - \ln \left( \frac{1}{2\sqrt{T}} \left( \frac{a_T}{T} \right)^{a_T} \left( \frac{b_T}{T} \right)^{b_T} \right) \\
& = \ln(2) + \frac{1}{2} \ln(T) - a_T \ln (p^*) - b_T \ln (1-p^*) \\
%& = \ln(2) + \frac{1}{2} \ln(T) + \sum_{t=1}^T \ell(p^*, q_t) \; .
& = \ln(2) + \frac{1}{2} \ln(T) + T \ln (2) -T \, \DKL\left(p^*\middle\|\frac{1}{2}\right) \; .
\end{align*}

% The following inequality will be useful
% $$
% \ln\left(1 + \beta g \right) \ge \left( \frac{1+g}{2} \right) \ln \left(1 + \beta\right) + \left( \frac{1-g}{2} \right) \ln \left(1 - \beta \right)
% \qquad \text{for $g \in [-1,1]$ and $\beta \in (-1,1)$}.
% $$


\subsection{From KT to Online Learning}

\begin{algorithm}[t]
  \begin{algorithmic}
  {
    %\STATE{\bfseries Parameters:} $\eta>0$
    \STATE{\bfseries Initialize:} $w_{1,i}=0$
    \FOR{$t=1,2,\dots$}
    \STATE{For each $i \in [d], \hat{p}_{t,i} = q_i |w_{t,i}|_+$}
    \STATE{Predict with $p_{t,i} = \frac{\hat{p}_{t,i}}{\norm{\hat{p}}_1}$}
    \STATE{Receive input vector $l_t \in [0,1]^d$ and suffer loss $\langle p_t, l_t\rangle$}
    \STATE{For each $i \in [d]$, set $g_{t,i} = \begin{cases}
      \langle p_t, \ell_t\rangle - \ell_{t,i} & \text{if } w_{t,i} \ge 0 \\
      |\langle p_t, \ell_t\rangle - \ell_{t,i}|_+ & \text{if } w_{t,i} < 0
      \end{cases} $}    
    \STATE{For each $i \in [d]$, set $w_{t+1,i} = \tfrac{\sum_{j=1}^{t} g_{j,i}}{t+1} \left(1+\sum_{j=1}^{t} w_{j,i} g_{j,i} \right)$}
    \ENDFOR
  }
  \end{algorithmic}
  \caption{Online Learning with Expert Advice based on \ac{KT}-betting.}
  \label{algo:kt_expert}
\end{algorithm}

The most interesting property for us is that the \ac{KT}-estimator can be used as a \ac{MBA}.
\begin{theorem}
\label{theo:kt_is_mba}
Define 
\begin{equation}
\label{eq:kt_potential}
f\left( x_{t}, \{|z_1|, \ldots, |z_{t}|\}\right) 
= \epsilon  \frac{2^t \cdot \Gamma \left(\frac{t+1}{2} + \frac{1}{2} x_{t} \right) \cdot \Gamma \left(\frac{t+1}{2} - \frac{1}{2} x_{t} \right)}{\pi \cdot t!}, 
\end{equation}
and $b_t= \frac{x_{t-1}}{t}$.
The couple $f,b_t$ satisfies the conditions in Assumption~1.
\end{theorem}

Hence, given the results in the previous two Sections, we can state the following Corollaries
\begin{cor}
  \label{cor:kt_hilbert}
  Let $g_t \in \fH$ an arbitrary sequence of vector, such that $\norm{g_t} \leq 1$ and define the vector $\theta_t=\sum_{i=1}^{t} g_i$.
  Define a vectorial algorithm that at each step outputs 
  \[
  w_t = \frac{\sum_{i=1}^{t-1} g_i}{t} \left(\epsilon+ \sum_{i=1}^{t-1} \langle w_i, g_i \rangle \right).
  \]
  Then the following holds
  \[
    Regret_T(u) \leq \norm{u} \sqrt{T \log \left(\frac{4 T^2 \norm{u}^2}{\epsilon^2} +1 \right)} +\epsilon\left(1-\frac{1}{2\sqrt{T}}\right) \quad \quad \quad \textnormal{ for all } u \in \fH \textnormal{ and } g_t \in \fH \; .
  \]
\end{cor}

\begin{cor}
\label{cor:kt_expert}
Let $l_t$ an arbitrary sequence of loss vectors in $[0,1]^d$. Using the notation of Algorithm~\ref{algo:kt_expert}, the following holds
\[
\sum_{t=1}^T \langle p_t, l_t\rangle -\sum_{t=1}^T \langle u , l_t\rangle 
\leq \sqrt{2 T \left( \frac{1}{2} \ln(T) + \ln(2) + \DKL(u,q)\right)} \; .
\]
\end{cor}