\appendix
\section{Appendix}

\begin{proof}[Proof of Lemma~\ref{lemma:kt}]
We prove the equality by induction $T$. For $T=0$ the equality is
$$
0 = - \ln \left( \frac{\Gamma(1/2) \cdot \Gamma(1/2)}{\pi \cdot 0!} \right)
$$
which holds true since $\Gamma(1/2) = \sqrt{\pi}$.
For $T \ge 1$, we use the induction hypothesis for $T-1$ and some algebraic manipulation
\begin{align*}
\sum_{t=1}^T \ell(p_t, q_t)
& = \ell(p_T, q_T) + \sum_{t=1}^{T-1} \ell(p_t, q_t) \\
& = \ell(p_T, q_T) - \ln \left( \frac{\Gamma(a_{T-1} + 1/2) \cdot \Gamma(b_{T-1} + 1/2)}{\pi \cdot (T-1)!} \right) \\
& = q_T \ln\left( \frac{1}{p_T}\right) + (1-q_T) \ln  \left( \frac{1}{1 - p_T} \right) - \ln \left( \frac{\Gamma(a_{T-1} + 1/2) \cdot \Gamma(b_{T-1} + 1/2)}{\pi \cdot (T-1)!} \right) \\
& = - q_T \ln\left( \frac{\frac{1}{2} + a_{T-1}}{T} \right) - (1-q_T) \ln\left( \frac{\frac{1}{2} + b_{T-1}}{T} \right) - \ln \left( \frac{\Gamma(a_{T-1} + 1/2) \cdot \Gamma(b_{T-1} + 1/2)}{\pi \cdot (T-1)!} \right) \\
& = - \ln\left( \left( \frac{\frac{1}{2} + a_{T-1}}{T} \right)^{q_T} \left( \frac{\frac{1}{2} + b_{T-1}}{T} \right)^{1-q_T} \frac{\Gamma(a_{T-1} + 1/2) \cdot \Gamma(b_{T-1} + 1/2)}{\pi \cdot (T-1)!} \right) \\
& = - \ln\left( \left( \frac{1}{2} + a_{T-1} \right)^{q_T} \left( \frac{1}{2} + b_{T-1} \right)^{1-q_T} \frac{\Gamma(a_{T-1} + 1/2) \cdot \Gamma(b_{T-1} + 1/2)}{\pi \cdot T!}  \right) \; .
\end{align*}
It remains to prove that expression inside logarithm equals
$$
\frac{\Gamma(a_{T} + 1/2) \cdot \Gamma(b_T + 1/2)}{\pi \cdot T!}
$$
We consider two cases. If $q_T = 1$ then
\begin{align*}
& \left( \frac{1}{2} + a_{T-1} \right)^{q_T} \left( \frac{1}{2} + b_{T-1} \right)^{1-q_T} \frac{\Gamma(a_{T-1} + 1/2) \cdot \Gamma(b_{T-1} + 1/2)}{\pi \cdot T!} \\
& = \left( \frac{1}{2} + a_{T-1} \right) \frac{\Gamma(a_{T-1} + 1/2) \cdot \Gamma(b_{T-1} + 1/2)}{\pi \cdot T!} \\
& = \frac{\Gamma(a_{T-1} + 3/2) \cdot \Gamma(b_{T-1} + 1/2)}{\pi \cdot T!} \\
& = \frac{\Gamma(a_{T} + 1/2) \cdot \Gamma(b_T + 1/2)}{\pi \cdot T!} \; ,
\end{align*}
where we have used that $\Gamma(x+1) = x \Gamma(x)$ for any real $x > 0$ and that $a_T = a_{T-1} + q_T = a_{T-1} + 1$ and $b_T = b_{T-1} + (1-q_T) = b_{T-1}$.
Similarly, if $q_T = 0$ then
\begin{align*}
& \left( \frac{1}{2} + a_{T-1} \right)^{q_T} \left( \frac{1}{2} + b_{T-1} \right)^{1-q_T} \frac{\Gamma(a_{T-1} + 1/2) \cdot \Gamma(b_{T-1} + 1/2)}{\pi \cdot T!} \\
& = \left( \frac{1}{2} + b_{T-1} \right) \frac{\Gamma(a_{T-1} + 1/2) \cdot \Gamma(b_{T-1} + 1/2)}{\pi \cdot T!} \\
& = \frac{\Gamma(a_{T-1} + 1/2) \cdot \Gamma(b_{T-1} + 3/2)}{\pi \cdot T!} \\
& = \frac{\Gamma(a_T + 1/2) \cdot \Gamma(b_T + 3/2)}{\pi \cdot T!} \; ,
\end{align*}
where we have used that $\Gamma(x+1) = x \Gamma(x)$ for any real $x > 0$ and that $a_T = a_{T-1} + q_T = a_{T-1}$ and $b_T = b_{T-1} + (1-q_T) = b_{T-1} + 1$.
\end{proof}

\begin{proof}[Proof of Theorem~\ref{theo:logloss}]
Let $\widetilde q_1, \widetilde q_2, \dots, \widetilde q_T$ independent
Bernoulli variables with parameters $q_1, q_2, \dots, q_T$ respectively.
Let
\begin{align*}
\widetilde a_t & = \sum_{i=1}^t \widetilde q_i  \; , \\
\widetilde b_t & = t - \widetilde a_t \; , \\
\widetilde p_t & = \frac{\frac{1}{2} + \sum_{i=1}^{t-1} \widetilde q_i}{t} = \frac{\frac{1}{2} + \widetilde a_{t-1}}{t} \; , \\
\widehat p & = \frac{\widetilde a_T}{T} = \frac{\widetilde a_T}{\widetilde a_T + \widetilde b_T} \; . \\
\end{align*}
Clearly,
\begin{align*}
q_i & = \Exp[\widetilde q_i] \; , \\
a_t & = \Exp[\widetilde a_t] \; , \\
b_t & = \Exp[\widetilde b_t] \; , \\
p_t & = \Exp[\widetilde p_t] \; , \\
p^* & = \Exp[\widehat p] \; .
\end{align*}

Since $\ell(p,q)$ is linear in $q$ and convex in $p$, we have
\begin{align*}
\sum_{t=1}^T \ell(p_t, q_t)
& = \sum_{t=1}^T \ell( \Exp[ \widetilde p_t], \Exp[\widetilde q_t]) \\
& = \Exp\left[ \sum_{t=1}^T \ell( \Exp[ \widetilde p_t], \widetilde q_t) \right] \\
& \le \Exp\left[ \sum_{t=1}^T \ell( \widetilde p_t, \widetilde q_t) \right] \\
& \le - \Exp\left[ \ln \left\{ \frac{1}{2\sqrt{T}} \left( \frac{\widetilde a_T}{T} \right)^{\widetilde a_T} \left( \frac{\widetilde b_T}{T} \right)^{\widetilde b_T} \right\} \right] \\
& = \ln(2) + \frac{1}{2} \ln(T) + \Exp \left[ - \widetilde a_T \ln \left( \frac{\widetilde a_T}{T} \right) - \widetilde b_T \ln \left( \frac{\widetilde b_T}{T} \right) \right] \\
& = \ln(2) + \frac{1}{2} \ln(T) + T \cdot \Exp \left[ H(\widehat p) \right] \\
& \le \ln(2) + \frac{1}{2} \ln(T) + T \cdot H(\Exp[\widehat p]) \\
& = \ln(2) + \frac{1}{2} \ln(T) + T \cdot H(p^*)~.
\end{align*}
\end{proof}

\begin{proof}[Proof of Lemma \ref{lemma:extremes}]
If $u$ or $v$ is zero, the inequality \eqref{equation:lemma-extremes-1} clearly holds. From now on we assume that
$u,v$ are non-zero. Let $c$ be the cosine of the angle of between $u$ and $v$.
More formally,
$$
c = \frac{\langle u, v \rangle}{\|u\| \cdot \|v\|} \; .
$$
With this notation, the left-hand side is
$$
\langle u, v \rangle + h(\|u + v\|) = c \|u\| \cdot \|v\|  + h(\sqrt{\|u\|^2 + \|v\|^2 + 2 c \|u\| \cdot \|v\|}) \; .
$$
We consider the last expression as a function of $c$ and we call it $f(c)$. The
inequality \eqref{equation:lemma-extremes-1} is equivalent to
$$
\forall c \in [-1,1] \qquad \qquad f(c) \ge \min \left\{f(+1), f(-1)\right\} \; .
$$
The last inequality is clearly true if $f:[-1,1] \to \R$ is concave. We now
check that $f$ is indeed concave, which we prove by showing that the second
derivative is non-positive. The first derivate of $f$ is
$$
f'(c) = \|u\| \cdot \|v\| + \frac{h'(\sqrt{\|u\|^2 + \|v\|^2 + 2 c \|u\| \cdot \|v\|}) \cdot \|u\| \cdot \|v\|}{\sqrt{\|u\|^2 + \|v\|^2 + 2 c \|u\| \cdot \|v\|}} \; .
$$
The second derivative of $f$ is
$$
f''(c) = \|u\|^2 \cdot \|v\|^2 \cdot \frac{h''(\sqrt{\|u\|^2 + \|v\|^2 + 2 c \|u\| \cdot \|v\|})  - \frac{1}{\sqrt{\|u\|^2 + \|v\|^2 + 2c \|u\| \cdot \|v\|}} \cdot h'(\sqrt{\|u\|^2 + \|v\|^2 + 2 c \|u\| \cdot \|v\|})  }{\|u\|^2 + \|v\|^2 + 2 c \|u\| \cdot \|v\|} \; .
$$
If we consider $x=\sqrt{\|u\|^2 + \|v\|^2 + 2 c \|u\| \cdot \|v\|}$, the
assumption $x \cdot h''(x) \le h'(x)$ implies that $f''(c)$ is non-positive.
This finishes the proof of the inequality \eqref{equation:lemma-extremes-1}.

Inequality \eqref{equation:lemma-extremes-2}, can be derived from inequality
\eqref{equation:lemma-extremes-1} as follows
\begin{align*}
\langle u, v \rangle + h(\|u + v\|)
& \ge \min_{u \in \H : \|u\| \le b} \min \left\{ \|u\| \cdot \|v\| + h(\|v\| + b), \ - \|u\| \cdot \|v\| + h(\|u\| - \|v\|) \right\} \\
& = \min_{z \in [0,b]} \min \left\{ z \cdot \|v\| + h(\|v\| + z), \ - z \cdot \|v\| + h(\|u\| - z) \right\} \\
& = \min_{z \in [-b,b]} z \cdot \|v\| + h(\|v\| + z) \\
& = \min \left\{ b \cdot \|v\| + h(\|v\| + b), \ - b \cdot \|v\| + h(\|u\| - b) \right\} \; .
\end{align*}
The inequality in the chain is the inequality \eqref{equation:lemma-extremes-1}.
The last equality follows since $g(z) = z \cdot \|v\| + h(\|v\| + z)$
is concave.
\end{proof}

\begin{proof}[Proof of Lemma \ref{lemma:gamma-function}]
First note that $f(x)$ is even, i.e., $f(x) = f(-x)$. Hence, odd terms of its
Maclaurin series are zero. The property $x \cdot f''(x) \ge f'(x)$ easily
follows if we show that the coefficients of the Maclaurin expansion of $f(x)$
around $0$ are non-negative, except possibly for the zero-order term $a_0$.
Indeed, if
$$
f(x) = \sum_{n=0}^\infty a_{2n} x^{2n} \qquad \text{and} \qquad a_2, a_4, a_6, \dots \ge 0
$$
then the condition $x \cdot f''(x) \ge f'(x)$ is equivalent to
$$
x \sum_{n=0}^\infty (2n)(2n-1) \cdot a_{2n} x^{2n-2} \ge \sum_{n=0}^\infty (2n) \cdot a_{2n} x^{2n-1} \; ,
$$
which holds for any $x \ge 0$ since each term on the right-hand side,
$(2n)(2n-1) a_{2n} x^{2n-1}$, is bigger or equal to the corresponding term on
the left-hand side, $(2n) a_{2n} x^{2n-1}$.

Since $\Gamma(x)$ is positive for any real $x > 0$, the function $f(x)$ is
positive on $(-a,a)$ and hence we can take its logarithm $g(x) = \ln(f(x))$.
Note that if $g(x)$ has non-negative (even) coefficients of its Maclaurin
expansion, except for possibly for the zero-order term, then the same holds for
$f(x) = \exp(g(x))$ since the coefficients of Maclaurin expansion $\exp(z) =
\sum_{n=0}^n \frac{z^n}{n!}$ are positive.

It thus remains to show that the coefficients of the Maclaurin expansion of
$g(x) = \ln(\Gamma(a+x) \Gamma(a-x))$ are non-negative. These coefficients can
be expressed in terms of \emph{logarithmic derivatives} of the Gamma function,
also called \emph{polygamma functions}. These are defined for any $n \ge 0$ as
any complex $x \in \C \setminus \N_{0}$ as
$$
\psi^{(n)}(x) = \frac{d^n\ln(\Gamma(x))}{dx^n} \; .
$$
The Mclaurin of expansion of $g(x)$ is
$$
g(x)
= \ln \left( \Gamma(a+x) \Gamma(a-x) \right)
= 2 \ln(\Gamma(a)) + \sum_{\substack{n \ge 2 \\ \text{$n$ even}}} \frac{\psi^{(n-1)}(a) \cdot x^n}{n!} \; .
$$
The fact that the coefficients $\psi^{(n-1)}(a)/n!$ are non-negative for even $n
\ge 2$ can be easily seen from the integral representation of polygamma
functions,
$$
\psi^{(n)}(z) = (-1)^{n+1} \int_0^\infty \frac{t^n e^{-zt}}{1-e^{-t}} dt \; ,
$$
valid for any $n \ge 1$ and any complex $z$ such that $\Re(z) > 0$.
\end{proof}

\begin{proof}[Proof of Theorem \ref{theo:wealth}]
We prove the inequality by induction on $T$. In the base case $T=0$, the
left-hand side is $\Wealth_0 = 1$, and the right-hand side equals $1$ as well
since $\sum_{t=1}^T g_t = 0$ and $\Gamma(\frac{1}{2}) = \sqrt{\pi}$. For $T \ge
1$, we will show that the difference of left and right side of the inequality is
non-negative. We have
\begin{align*}
& \Wealth_T \ - \ \frac{2^T \cdot \Gamma \left(\frac{T+1}{2} + \frac{1}{2}\left\|\sum_{t=1}^T g_t \right\| \right) \cdot \Gamma \left(\frac{T+1}{2} - \frac{1}{2} \left\|\sum_{t=1}^T g_t \right\| \right)}{\pi \cdot T!} \\
& = (1 + \langle g_T, \beta_T \rangle) \Wealth_{T-1} \ - \ \frac{2^T \cdot \Gamma \left(\frac{T+1}{2} + \frac{1}{2}\left\|\sum_{t=1}^T g_t \right\| \right) \cdot \Gamma \left(\frac{T+1}{2} - \frac{1}{2} \left\|\sum_{t=1}^T g_t \right\| \right)}{\pi \cdot T!} \\
& = \Wealth_{T-1} \ + \ \frac{1}{T} \cdot \Wealth_{T-1} \cdot \left(\langle g_T, T \beta_T \rangle - \frac{2^T T \cdot \Gamma \left(\frac{T+1}{2} + \frac{1}{2}\left\|\sum_{t=1}^T g_t \right\| \right) \cdot \Gamma \left(\frac{T+1}{2} - \frac{1}{2} \left\|\sum_{t=1}^T g_t \right\| \right)}{\Wealth_{T-1} \cdot \pi \cdot T!} \right) \\
& = \Wealth_{T-1} \ + \ \frac{1}{T} \cdot \Wealth_{T-1} \cdot \left( \left\langle g_T, \sum_{t=1}^{T-1} g_t \right\rangle + h\left( \left\|g_T + \sum_{t=1}^{T-1} g_t \right\| \right) \right) \; .
\end{align*}
where we define
\begin{align*}
h(x) & = - C \cdot \Gamma \left(\frac{T+1}{2} + \frac{x}{2} \right) \cdot \Gamma \left(\frac{T+1}{2} - \frac{x}{2} \right) \; ,  \\
C & = \frac{2^T \cdot T}{\Wealth_{T-1} \cdot \pi \cdot T!} \; .
\end{align*}
We apply Lemma~\ref{lemma:extremes} to $h(x)$. First note that $C > 0$. Second,
$\Gamma(x)$ is logarithmically convex on $[0,\infty)$, and hence $h(x)$ is
logarithmically concave and hence concave. Third, $h(x)$ is clearly even and its
defined on an interval $(-\frac{T+1}{2}, \frac{T+1}{2})$. To see that $h(x)$
satisfies the condition $x \cdot h''(x) \le h'(x)$, we apply
Lemma~\ref{lemma:gamma-function}. Let $f(x) = \Gamma(\frac{T+1}{2}+x)
\Gamma(\frac{T+1}{2}-x)$. By Lemma~\ref{lemma:gamma-function}, we have $x \cdot
f''(x) \ge f(x)$ for all $x \in [0,a)$. Since $h(x) = - C f(x/2)$, we have $x
\cdot h''(x) \le h'(x)$ for $x \in [0,T+1)$.

Therefore, since $\|g_t\| \le 1$ and $\|\sum_{t=1}^{T-1} g_t\| < T$,
\begin{align*}
& \Wealth_{T-1} \ + \ \frac{1}{T} \cdot \Wealth_{T-1} \cdot \left( \left\langle g_T, \sum_{t=1}^{T-1} g_t \right\rangle + h\left( \left\|g_T + \sum_{t=1}^{T-1} g_t \right\| \right) \right) \\
& \ge \Wealth_{T-1} \ + \ \frac{1}{T} \cdot \Wealth_{T-1} \cdot \min \left\{ \left\| \sum_{t=1}^{T-1} g_t \right\| + h\left( \left\|\sum_{t=1}^{T-1} g_t \right\| + 1 \right), - \left\| \sum_{t=1}^{T-1} g_t \right\| + h\left(\left\|\sum_{t=1}^{T-1} g_t \right\| - 1 \right) \right\} \\
& = \Wealth_{T-1} \ + \ \Wealth_{T-1} \cdot \min \left\{ \| \beta_T \| + \frac{1}{T} \cdot h\left( \left\|\sum_{t=1}^{T-1} g_t \right\| + 1 \right), - \|\beta_T\| + \frac{1}{T} \cdot h\left(\left\|\sum_{t=1}^{T-1} g_t \right\| - 1 \right) \right\} \\
& = \min \left\{ \Wealth_{T-1} (1 + \| \beta_T \|) + \frac{\Wealth_{T-1}}{T} h\left( \left\|\sum_{t=1}^{T-1} g_t \right\| + 1 \right), \Wealth_{T-1} (1 - \| \beta_T \|) + \frac{\Wealth_{T-1}}{T} h\left(\left\|\sum_{t=1}^{T-1} g_t \right\| - 1 \right) \right\}
\end{align*}
It remains to prove that the last expression is non-negative, which we do by
proving that each of the two sub-expressions of the minima are non-negative. The
first sub-expression is
\begin{align*}
& \Wealth_{T-1} (1 + \| \beta_T \|) + \frac{\Wealth_{T-1}}{T} h\left( \left\|\sum_{t=1}^{T-1} g_t \right\| + 1 \right) \\
& =  \Wealth_{T-1} (1 + \|\beta_T\|) - \frac{2^T \cdot \Gamma \left(\frac{T+1}{2} + \frac{1}{2}\left\|\sum_{t=1}^{T-1} g_t \right\| + \frac{1}{2} \right) \cdot \Gamma \left(\frac{T+1}{2} - \frac{1}{2} \left\|\sum_{t=1}^{T-1} g_t \right\| - \frac{1}{2} \right)}{\pi \cdot T!} \\
& =  \Wealth_{T-1} (1 + \|\beta_T\|) - \frac{2^T \left( \frac{T}{2} + \frac{1}{2} \left\|\sum_{t=1}^{T-1} g_t \right\| \right) \Gamma \left(\frac{T}{2} + \frac{1}{2}\left\|\sum_{t=1}^{T-1} g_t \right\| \right) \cdot \Gamma \left(\frac{T}{2} - \frac{1}{2} \left\|\sum_{t=1}^{T-1} g_t \right\| \right)}{\pi \cdot T!} \\
& =  \Wealth_{T-1} (1 + \|\beta_T\|) - \frac{2^{T-1} \left( 1 + \|\beta_T\| \right) \Gamma \left(\frac{T}{2} + \frac{1}{2}\left\|\sum_{t=1}^{T-1} g_t \right\| \right) \cdot \Gamma \left(\frac{T}{2} - \frac{1}{2} \left\|\sum_{t=1}^{T-1} g_t \right\| \right)}{\pi \cdot (T-1)!} \\
& =  (1 + \|\beta_T\|) \left( \Wealth_{T-1} - \frac{2^{T-1} \Gamma \left(\frac{T}{2} + \frac{1}{2}\left\|\sum_{t=1}^{T-1} g_t \right\| \right) \cdot \Gamma \left(\frac{T}{2} - \frac{1}{2} \left\|\sum_{t=1}^{T-1} g_t \right\| \right)}{\pi \cdot (T-1)!} \right) \; .
\end{align*}
The second sub-expression is
\begin{align*}
& \Wealth_{T-1} (1 - \| \beta_T \|) + \frac{\Wealth_{T-1}}{T} h\left(\left\|\sum_{t=1}^{T-1} g_t \right\| - 1 \right) \\
& = \Wealth_{T-1} (1 - \|\beta_T\|) - \frac{2^T \cdot \Gamma \left(\frac{T+1}{2} + \frac{1}{2}\left\|\sum_{t=1}^{T-1} g_t \right\| - \frac{1}{2} \right) \cdot \Gamma \left(\frac{T+1}{2} - \frac{1}{2} \left\|\sum_{t=1}^{T-1} g_t \right\| + \frac{1}{2} \right)}{\pi \cdot T!} \\
& = \Wealth_{T-1} (1 - \|\beta_T\|) - \frac{2^T \left( \frac{T}{2} - \frac{1}{2}\|\sum_{t=1}^{T-1} g_t\| \right) \cdot \Gamma \left(\frac{T}{2} + \frac{1}{2}\left\|\sum_{t=1}^{T-1} g_t \right\| \right) \cdot \Gamma \left(\frac{T}{2} - \frac{1}{2} \left\|\sum_{t=1}^{T-1} g_t \right\| \right)}{\pi \cdot T!} \\
& = \Wealth_{T-1} (1 - \|\beta_T\|) - \frac{2^{T-1} \left( 1 - \|\beta_T\| \right) \cdot \Gamma \left(\frac{T}{2} + \frac{1}{2}\left\|\sum_{t=1}^{T-1} g_t \right\| \right) \cdot \Gamma \left(\frac{T}{2} - \frac{1}{2} \left\|\sum_{t=1}^{T-1} g_t \right\| \right)}{\pi \cdot (T-1)!} \\
& = (1 - \|\beta_T\|) \left( \Wealth_{T-1}  - \frac{2^{T-1} \Gamma \left(\frac{T}{2} + \frac{1}{2}\left\|\sum_{t=1}^{T-1} g_t \right\| \right) \cdot \Gamma \left(\frac{T}{2} - \frac{1}{2} \left\|\sum_{t=1}^{T-1} g_t \right\| \right)}{\pi \cdot (T-1)!} \right) \; .
\end{align*}
In both cases we have used $\Gamma(x+1) = x \cdot \Gamma(x)$ which holds for any
real $x > 0$. Induction hypothesis and $\|\beta_T\| \in (-1,1)$ imply
that both sub-expressions are non-negative.
\end{proof}